%\linebreak
\documentclass[a4paper,12pt]{report}
\usepackage[utf8x]{inputenc}
\usepackage[usenames,dvipsnames,svgnames,table]{xcolor}
\usepackage{fancyvrb}
\usepackage{float}
\usepackage{booktabs}
\usepackage{multirow}

\usepackage{titlesec}

\usepackage[round]{natbib}
\bibliographystyle{dinat}
\usepackage{alltt}
\usepackage{adjustbox}
\usepackage[usenames,dvipsnames,svgnames,table]{xcolor}
\usepackage[T1]{fontenc}
\usepackage[francais]{babel}
\usepackage[top=1cm,bottom=2.5cm,left=2cm,right=2cm]{geometry}
\usepackage{fancyhdr}
\usepackage{graphics}
\usepackage{graphicx}
\usepackage{color}
\usepackage{colortbl}
\usepackage{url}
\usepackage{amsfonts}
\usepackage{amsmath}
\usepackage[french]{minitoc}
\usepackage[colorlinks]{hyperref}
\usepackage[gen]{eurosym}
\usepackage{caption}
\usepackage[nottoc, notlof, notlot]{tocbibind}
\usepackage{pdfpages}
\usepackage[Lenny]{fncychap}
\usepackage{fancyhdr}
\usepackage{amsmath}
\usepackage{epigraph}
\usepackage{setspace}
\linespread{1.5}
\usepackage{booktabs}
\usepackage{makecell}
\usepackage{tabularx}

\usepackage{lscape}

\usepackage{rotating}
\hypersetup{	
	colorlinks=true, 
	breaklinks=true, 
	urlcolor= blue, 
	linkcolor= black,	
	citecolor=blue,	 
	pdftitle={},
	pdfauthor={},  
	pdfsubject={},	 
	pdfcreator={}
}

\usepackage[T1]{fontenc}

\usepackage{subcaption}

\setcounter{page}{1}
\fancyhead[L]{\leftmark}
\fancyhead[R]{}

\fancyfoot[C]{}
\fancyfoot[R]{\thepage}

\setcounter{tocdepth}{4}
\setcounter{secnumdepth}{4}

\makeatletter \@addtoreset{chapter}{part}
\makeatother
\begin{document}
	

	
		\newpage
		\begin{center}
			\includegraphics[width=0.5\linewidth]{images/u-paris10.png}
			\vspace{3cm}
		\end{center}
			
		\begin{center}
			\vspace{0.3cm}
			\large  
			\textbf{Mémoire Master 2 MIAGE classique}\\
			\vspace{0.7cm} \large Thème\\
			\vskip 0.2in
		\end{center}
		

		\begin{center}
        	\hrule
        \vspace{.7cm}
        {\Huge\textbf{ Etude des approches de modélisation des entrepôts de données (DataWarehouse) }}\\
        \vspace{.7cm}
            \hrule
		\end{center}			
		
			\vspace{1cm}
		\normalsize
		\vskip 0.2in
		
	
		~~\\
		\begin{flushleft}
			\begin{center}
				\begin{tabular}{lllllllllll}
					\textbf{Réalisé par :}    &&&&&&&&&& \textbf{Encadré par :}\\
					\hspace{0.9cm} Zakaria MEDJIR    &&&&&&&&&& \hspace{0.2cm} Monsieur  Reda BENDRAOU  \\
					\hspace{5cm}             &&&&&&&&&&  
					
				\end{tabular}
			\end{center}
		\end{flushleft}
		
		\vskip 0.1in
		
		\begin{tabular}{llllllllll}
	
		\end{tabular}
		
		~~\\
		\vskip 0.3in
		\begin{center}
			\vspace{0.3cm} \textbf{Promotion : 2017-2018}
		\end{center}
	\newpage
		\pagenumbering{roman}
		
		
		
		
		\section*{Introduction}
\vspace{1cm}
    



\textcolor{black}{Aujourd’hui,parmis les défis auxquels font face les entreprises, ont rouve l’exploitation et l’analyse de données opérationnelles qu’elles détiennent dans leurs sources de données hétérogènes. Le but ultime de ces tâches est d’obtenir de l’information utile pour la prise de décision.
}

 \textcolor{black}{La Business intelligence ou système décisionnel est l’anneau manquant qui peut transformer ces données brutes en informations utiles et pertinentes qui peuvent supporter les décisions prises par les dirigeants des entreprises.Un concept central dans un tel système est l’entrepôt de données(Datawarehouse).Ce dernier est donc un composant principal du système décisionnel qui a pour but de stocker les données opérationnelles, provenant de plusieurs sources,dans une perspective décisionnelle et de les fournir auxu tilisateurs sous certaines formes pour des fins d’analyse.
 }


\textcolor{black}{La mise en place d’un entrepôt de données nécessite une approche de modélisation qui prend en considération tous les aspects dedéveloppement comme la modélisation de données, la gestion de projet,la gestion des risques,le déploiement et bien d’autres aspects essentiels.
Depuis des années,deux approches s’affrontent quant à lamodélisation des entrepôts de données:l’approche de modélisation par sujet d’Inmon et l’approche dimensionnelle de Kimball.Cependant,ces dernières années,une troisième approche est apparue et elle gagne du terrain d’année en année.Cette approche est créée par Linstedt et s’appelle«Data Vault».
Dans le cadre de ce mémoire de Master,nous élaborons une étude comparatives de ces trois approches de modélisation d’entrepôt de données.}


\textcolor{black}{La mise en place d’un entrepôt de données nécessite une approche de modélisation qui
prend en considération tous les aspects de développement comme la modélisation de données, la
gestion de projet, la gestion des risques, le déploiement et bien d’autres aspects essentiels.}

\textcolor{black}{ Depuis des années, deux approches s’affrontent quant à la modélisation des entrepôts de
données : l’approche de modélisation par sujet  \textbf{d’Inmon }et l’approche dimensionnelle de  \textbf{ Kimball}.
Cependant, une troisième approche est apparue et elle gagne du terrain d’année en année. Cette
approche est créée par Linstedt et s’appelle \textbf{ « Data Vault »}}

\textcolor{black}{Le but de ce travail est de faire une synthèse sur la BI à travers ces trois approches. D’abord, je ferai une étude approfondie sur les systèmes décisionnels et l’architecture autour de laquelle ils sont construits, en détaillant chacun de ses composants et en mettant l’accent sur l’entrepôt de données. Ensuite je vais me focaliser sur la présentation de chaque approche : sa définition, sa philosophie, son architecture ainsi que sa méthodologie de développement. Enfin, je dresserai une analyse comparative entre les trois approches en s’appuyant sur des critères tels que :}

\begin{itemize}
	\item La méthodologie et l'architecture : structure \& architecture, méthodologie de développement , coût et temps de déploiement... etc
	\item La modélisation de données : orientation des données, les outils utilisés, l'implication de l'utilisateur final ... etc 
	\item Intégration de données : intégration des sources multiples, complexité de processus ETL ... etc
	\item Management du cycle du vie : facilité du changement du modèle, performances d'interrogation … etc
	
	
\end{itemize}



\textcolor{black}{ Cette étude comparative sera accompagnée par des tests sur des données réelles (au sein de l'entreprise de mon contrat professionalisation) avec un groupe témoin afin d’aboutir à une synthèse permettant de répondre à la question :  \textbf{ « Quelle approche dans quelle situation ?».}}
\\
	
\vspace{1cm}
\section*{Bibliographie}
\vspace{1cm}



	\item[] \textcolor{black}{
\textbf{[Weir2008]}  Weir : A Configuration Approach for Selecting a Data Warehouse Architecture, Thesis,2008}
\\

	\item[] \textcolor{black}{
\textbf{(Kimball et Ross, 2013)} Kimball, Ralph ; Ross, Margy : The data warehouse toolkit : The definitive guide to dimensional modeling. John Wiley \& Sons, 2013.	
}
\\

	\item[] \textcolor{black}{
\textbf{[Denis 2008]} Denis : Conception et réalisation d’un entrepôt de données institutionnel dans une perspective de support à la prise de décision, Thesis,2008.}
\\

	\item[] \textcolor{black}{
\textbf{[Teste2009]} Teste, Olivier: Modélisation et manipulation des systèmes OLAP:de l’intégration des documents à l’usager, Université Paul Sabatier-Toulouse III, Dissertaion, 2009}


	\item[] \textcolor{black}{
\textbf{[Adamson2012] } Adamson, Christopher : Mastering data warehouse aggregates: solutions for star schema performance. JohnWiley&Sons ,2012.}
\\

	\item[] \textcolor{black}{
\textbf{[Awel2014]}  Awel : Data Vault Modelling, Thesis,2014}
\\

	\item[] \textcolor{black}{
\textbf{[Awel2014]} Mathiews, Diane: Data vault et bi. URL:http://fr.slideshare.net/dlinstedt/prsentation-data-vault-et-bi-v20120508}
\\

	\item[] \textcolor{black}{
\textbf{[Orlov2014] } Orlov, Vadim: Data Warehouse Architecture : InmonCIF, Kimball Dimensional or Linstedt DataVault ? 2014. –URL: https://blog.westmonroepartners.com/data-warehouse-architecture-inmon-cif-kimball-dimensional-or-linstedt-data-vault/}
\\

	\item[] \textcolor{black}{
\textbf{(Poletto, 2012)} Poletto, Maxime : L’informatique décisionnelle, These professionnelle, 2012. – URL http://news.exia.cesi.fr/wp-content/uploads/2012/06/
Maxime-Poletto-Th\%C3\%A8se.pdf.}


\end{document}